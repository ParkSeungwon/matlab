\documentclass[12pt,a4paper]{article}
\synctex=1
\usepackage[utf8]{inputenc}
\usepackage[margin=1cm]{geometry}
\usepackage{graphicx}
%\usepackage{verbatim}
\usepackage{amsmath}
\usepackage{amsfonts}
\usepackage{amssymb}
\usepackage{listings}
\usepackage{enumitem}
\usepackage{textcomp}
\usepackage{courier}
\usepackage{libertine}
\usepackage{pgfornament}
\usepackage{eso-pic}
\usepackage[hangul]{kotex}
\linespread{1.3}

\title{
	\centering
	\pgfornament[width=12cm,color=teal]{84}\\
	\vspace{1cm}
	\fontsize{50}{50} \selectfont {정보통신 수학 및 실습\\Lab assignment}\\
		\pgfornament[width=12cm,color=teal]{88}\\
	\vfill}
\author{
	\LARGE
	\begin{tabular}{rcc}
		\hline
		학번 : & 2016110056 & 2012112130\\ 
		이름 : & 박승원 & 노희승\\
		편성 : & 20조 & \today\\
		\hline
	\end{tabular}\vspace{1cm}
	\\
\includegraphics[width=0.5\textwidth]{logo.jpg}
	}
\date{}

\begin{document}
\maketitle
\pagenumbering{gobble}
\noindent
\lstset{language=matlab, columns=flexible, tabsize=4, frame=shadowbox, showstringspaces=false, breaklines=true, upquote=true, basicstyle=\normalsize}

\renewcommand{\thesubsubsection}{\alph{subsubsection})}
\renewcommand{\thesubsection}{\arabic{subsection}.}
\newpage
\section*{Chapter 13 Lab Assignment(DFT \& FFT)}

\subsection{Let x(t) be the following function:}

\[
x(t)=\left\{
\begin{array}{ll}
1 &0\leq t \leq \frac{\pi}{2}\\
\cos(6\pi - t -\frac{\pi}{2} ) & \frac{\pi}{2}\leq t \leq \frac{3\pi}{2}\\
-1 &\frac{3\pi}{2}\leq t \leq 2\pi
\end{array}
\right.
\]

\subsubsection{Plot x(t) where t=[0:0.001:2*pi].} 
\subsubsection{Plot abs(fft(x(t)) where x-axis is labeled as radian.} 
\subsubsection{Let the sampling frequency is 30 Hz.  Plot x(nTs) where $0 < t < 2ㅠ$.} 
\subsubsection{Plot abs(fft(x(nTs)) where x-axis is labeled as radian.} 
\subsubsection{Let the sampling frequency is 5 Hz.  Plot x(nTs) where $0 < t < 2ㅠ$.} 
\subsubsection{Plot abs(fft(x(nTs)) where x-axis is labeled as radian.} 
\subsubsection{Compare the results of b, d, f and describe your findings.} 
\subsection{Answer the following questions about x(n) where x(n)= (0.3)n, n = 0, 1, 2, and 3 and x(n)=0 otherwise.}  
	
\subsubsection{Program 4-points fft of x(n).} 
\subsubsection{Find DFT of x(n) using fft() and compare the result of (a)} 
\end{document}
