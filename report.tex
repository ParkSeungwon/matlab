\documentclass[12pt,a4paper]{article}
\synctex=1
\usepackage[utf8]{inputenc}
\usepackage[margin=1cm]{geometry}
\usepackage{graphicx}
%\usepackage{verbatim}
\usepackage{amsmath}
\usepackage{amsfonts}
\usepackage{amssymb}
\usepackage{listings}
\usepackage{enumitem}
\usepackage{textcomp}
\usepackage{courier}
\usepackage{libertine}
\usepackage{pgfornament}
\usepackage{eso-pic}
\usepackage[hangul]{kotex}
\linespread{1.3}

\title{
	\centering
	\pgfornament[width=12cm,color=teal]{84}\\
	\vspace{1cm}
	\fontsize{50}{50} \selectfont {정보통신 수학 및 실습\\Homework}\\
		\pgfornament[width=12cm,color=teal]{88}\\
	\vfill}
\author{
	\LARGE
	\begin{tabular}{rl}
		\hline
		학번 : & 2016110056\\ 
		학과 : & 불교학부 \\
		이름 : & 박승원\\
		날짜 : & \today\\
		\hline
	\end{tabular}\vspace{2cm}
	\\
\includegraphics[width=0.5\textwidth]{logo.jpg}
	}
\date{}


\begin{document}
\maketitle
\pagenumbering{gobble}
\noindent
\lstset{language=matlab, columns=flexible, tabsize=4, frame=shadowbox, showstringspaces=false, breaklines=true, upquote=true, basicstyle=\normalsize}

\renewcommand{\thesubsubsection}{\alph{subsubsection})}
\renewcommand{\thesubsection}{\arabic{subsection}.}
\newpage
\section*{Chapter 5 Homework}
\subsection{Find the derivatives of the following functions at x1.} 
\subsubsection{$y= 3x^2+2x-1, x1=2$}  
$y' = 6x + 2\\ y'(2) = 14$
\subsubsection{$y= 3x+7, x1=-1$}
3
\subsection{Find the derivatives of the following functions and their integration parts of the fundamental theorem of calculus.} 
\subsubsection{$\cos x\sin x$} 
$-\sin^2 x+\cos^2 x$
\subsubsection{$\log(10x)$}
$(1 + \log x)' = \dfrac{1}{x}$
\subsubsection{$e^{-3x}$} 
$-3e^{-3x}$

\subsection{Find the values of the following integrations.} 
\subsubsection{$\displaystyle\int_{-1}^1 x^2+3x+2 dx$}
$\left[ \dfrac{1}{3}x^3+6x+2x \right]_{-1}^1 = \dfrac{2}{3}$
\subsubsection{$\displaystyle\int_{-1}^1 e^{2x}+e^x dx$}
$\left[ \dfrac{1}{2}e^{2x}+e^x \right]_{-1}^1=- \dfrac{1}{e} - \dfrac{1}{2 e^{2}} + e + \dfrac{e^{2}}{2}$
\subsubsection{$\displaystyle\int_1^2 \dfrac{1}{2} dx$}
$\left[ \dfrac{1}{2} x \right]_1^2=\dfrac{1}{2}$
\subsection{Compute the following indefinite integrations.} 
\subsubsection{$\displaystyle\int  xe^x dx$}
$(xe^x)'=e^x + xe^x$에서 양변을 적분하면
$xe^x = e^x + \displaystyle\int xe^x dx\\
\therefore \displaystyle\int xe^x dx = xe^x - e^x$
\subsubsection{$\displaystyle\int x\sin x dx$}
$(x\cos x)' = \cos x - x\sin x$에서 양변을 적분하면
$x\cos x = \sin x - \displaystyle\int x\sin x dx\\
\therefore \displaystyle\int x\sin x dx = \sin x - x\cos x$
\subsubsection{$\displaystyle\int \dfrac{\log x}{x} dx$}
$(\log^2 x)' = \dfrac{2 \log x}{x}\\ 
\therefore \displaystyle\int \dfrac{\log x}{x}= \dfrac{1}{2} \log^2 x $
\subsubsection{$\displaystyle\int  e^{-x}\cos x dx$}
$(e^{-x}\sin x)' = -e^{-x}\sin x + e^{-x}\cos x\\
(e^{-x}\cos x)'= -e^{-x}\cos x -e^{-x}\sin x\\$
위 식에서 아래 식을 빼면 \\
$(e^{-x}\sin x)'- (e^{-x}\cos x)' = 2e^{-x}\cos x\\$
양변을 적분하면 
$\therefore \displaystyle\int e^{-x}\cos x = \dfrac{1}{2}(e^{-x}\sin x -e^{-x}\cos x)$
\subsection{Find the Taylor series of the following functions.}  
무한히 미분 가능한 임의의 함수를 무한한 차수의 다항식 
$C_nx^n+C_{n-1}x^{n-1}+C_{n-2}x^{n-2}+ ... C_2x^2+C_1x + C_0$으로 나타낼 수 있다.($n \rightarrow \infty$)
연립방정식을 생각해보면 두 개의 점이 있으면 그 것을 나타내는 일차식을 결정하는 계수를 구할 수 있다. 이 때에 차수는 일차식이면 된다. 점의 갯수가 많아질 수록 다항식을 차수를 늘리면 그에 상응하는 계수를 구할 수 있다. 이를 확장하면 무한의 차수의 방정식으로 유한한 갯수의 점을 가지는 함수를 구할 수 있다. 테일러 식은 근사식이므로 유한한 갯수의 점을 가지는 함수를 가정해도 된다. 그러므로 임의의 식을 \\
$f(x) = C_nx^n+C_{n-1}x^{n-1}+C_{n-2}x^{n-2}+ \cdots +C_2x^2+C_1x + C_0$로 나타낼 수 있다.\\
다음으로 계수 $C_n$를 함수 f를 n번 미분하여 0을 대입한 것으로 구할 수 있다. 자신보다 큰 항의 계수는 0의 대입으로 인해 사라지고, 자신보다 차수가 작은 항은 미분으로 사라진다.\\
$C_n = \dfrac{f^n(0)}{n!}$\\
그러므로, $f(x)=\displaystyle\sum_{k=0}^n \dfrac{f^k(0)}{k!}x^k \\
= f(0) + f'(0)x+ \dfrac{f''(0)}{2}x^2 + \dfrac{f'''(0)}{6}x^3 \cdots$\\
이 때에 x의 크기가 1보다 작으면, 자승에 의해서 $x^n$이 작아지므로, 테일러 급수를 통해 어떠한 식의 근사치를 구해 갈 수 있다.
\subsubsection{$e^{2x}$}
$1 + 2x + \cdots + \dfrac{2^n}{n!}x^n = \displaystyle\sum_{k=0}^n\dfrac{2^k}{k!}x^k$
\subsubsection{$\sin{2x}$}
짝수번 미분하면 0을 대입시 0이 되므로 홀수번 미분시만 고려하면 된다.\\
$\displaystyle\sum_{k=0}^n(-1)^k\dfrac{2^{2k+1}}{(2k+1)!}x^{2k+1}$
\subsubsection{$\cos{2x}$}
홀수번 미분하면 0을 대입시 0이 되고, 짝수번 미분시 0을 대입하면 1이 되므로 짝수번 미분만 고려하면 된다.\\
$\displaystyle\sum_{k=0}^{n}(-1)^{k}\dfrac{2^{2k}}{(2k)!}x^{2k}$
\end{document}
