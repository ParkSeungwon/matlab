\documentclass[12pt,a4paper]{article}
\synctex=1
\usepackage[utf8]{inputenc}
\usepackage[margin=1cm]{geometry}
\usepackage{graphicx}
%\usepackage{verbatim}
\usepackage{amsmath}
\usepackage{amsfonts}
\usepackage{amssymb}
\usepackage{listings}
\usepackage{enumitem}
\usepackage{textcomp}
\usepackage{courier}
\usepackage{libertine}
\usepackage{pgfornament}
\usepackage{eso-pic}
\usepackage[hangul]{kotex}
\linespread{1.3}

\title{
	\centering
	\pgfornament[width=12cm,color=teal]{84}\\
	\vspace{1cm}
	\fontsize{50}{50} \selectfont {정보통신 수학 및 실습\\Summary}\\
		\pgfornament[width=12cm,color=teal]{88}\\
	\vfill}
\author{
	\LARGE
	\begin{tabular}{rl}
		\hline
		학번 : & 2016110056\\ 
		학과 : & 불교학부 \\
		이름 : & 박승원\\
		날짜 : & \today\\
		\hline
	\end{tabular}\vspace{2cm}
	\\
\includegraphics[width=0.5\textwidth]{logo.jpg}
	}
\date{}


\begin{document}
\maketitle
\pagenumbering{gobble}
\noindent
\lstset{language=matlab, columns=flexible, tabsize=4, frame=shadowbox, showstringspaces=false, breaklines=true, upquote=true, basicstyle=\normalsize}

\renewcommand{\thesubsubsection}{\alph{subsubsection})}
\renewcommand{\thesubsection}{\arabic{subsection}.}
\newpage

\section{미분방정식}
\subsection{linear}
$f(x_1+x_2)=f(x_1)+f(x_2)$와 $
f(kx)=kf(x)$를 만족하면 linear하다고 한다.
\subsection{homogeneous 제차, 비제차}
$y'+p(x)y=r(x)$에서 r(x)가 0이면 homogeneous

\subsection{Laplace transform}
\begin{gather*}
L(s)=\int_0^\infty f(t)e^{-st}dt
\end{gather*} 
\subsection{Fourier Series}
주기 함수의 주기가 $f_0$일 때 이를 다음과 같은 주기의 n배가 되는 항들의 합으로 나타낼 수 있다.
\begin{gather*}
f(t)=\sum_{n=-\infty}^{\infty} c_n e^{-2\pi f_0 nt}\\
c_n = \frac{1}{T}\int_{-\frac{T}{2}}^{\frac{T}{2}}f(t)e^{-2\pi f_0 nt}dt\\
c_{-k} = \bar{c_k}
\end{gather*} 
\subsection{Fourier transform}
비주기 함수의 경우 주기함수의 주기가 무한대인 것으로 생각할 수 있다.
\begin{gather*}
F(w) = \int_{-\infty}^{\infty}f(t)e^{-jwt}dt\\
f(t) = \int_{-\infty}^{\infty}F(w)e^{jwt}dw\\
g_1(x)*g_2(x) = \int_{-\infty}^{\infty}g_1(\tau)g_2(x-\tau)d\tau\\
F(g_1(x)*g_2(x)) = F(g_1(x))\cdot F(g_2(x)) = G_1(f)\cdot G_2(f)
\end{gather*}
\subsection{Discrete Time Fourier Transform}
\begin{gather*}
X(w) = \sum_{n=-\infty}^{\infty}x[n]e^{-jwn}\\
x[n] = \frac{1}{2\pi}\int_{-\pi}^{\pi}X(w)e^{jwn}dw
\end{gather*}
\end{document}

